No contexto de IoT, uma tecnologia relevante é o padrão OSGI. Este padrão define um mecanismo que oferece suporte a plug-ins em Java, controlando sua instalação e o seu ciclo de vida. Os plug-ins criados em um servidor OSGI podem oferecer e consumir serviços entre si, bem como se comunicar \cite{IntroducaoOsgi}. Tais plug-ins podem construir aplicações que podem ser quebradas em vários módulos e que permita facilmente, gerenciar os cruzamentos entre eles. Gerenciar os módulos de forma independente, ou gerenciar um grupo de módulos que trabalham juntos \cite{OsgiVantagens}.

O padrão OSGI pode ser utilizado para o desenvolvimento de aplicações baseados em Internet das coisas e o Kura é um exemplo de ferramenta que dá suporte a tal desenvolvimento. Sendo um servidor de aplicações OSGI, o kura permite a criação de plug-ins usando a linguagem de programação java. Também permite o gerenciamento do ciclo de vida de plug-ins, e teste das soluções desenvolvidas usando um emulador que permite a execução em um ambiente de desenvolvimento. O kura também oferece serviços para conexão com portas seriais, GPS, bluetooth e nuvem.

O Kura é uma das ferramentas, baseada na tecnologia OSGI, que dá suporte ao desenvolvimento de soluções para a internet das coisas. O Kura se trata de um servidor OSGI com foco em IoT, que fornece uma plataforma de suporte a plug-ins já com vários plugins pré-definidos que são úteis na criação de sistemas para IoT.

O kura se encontra disponível para os sistemas operacionais Mac e Linux. O Linux é ao mesmo tempo um kernel (ou núcleo) e um sistema operacional que roda sobre ele. \cite{campos2006linux}. O Linux possui alguns programas interessantes tais como fswebcam e espeak. O espeak se trata de um sintetizador de voz para o Ubuntu, onde o usuário pode inserir entradas de textos e estas serão sintetizadas e reproduzidas pelo espeak. O espeak dá suporte a diversas línguas, inclusive o português. Outro programa disponível para o linux é o fswebcam, que se trata de um programa para tirar fotos. O programa recebe diversos parâmetros antes de tirar a foto, tais como informações sobre o tamanho do arquivo, formato, e tempo que o programa espera antes de tirar a foto. O local onde o arquivo será salvo, também pode ser passado como parâmetro.


