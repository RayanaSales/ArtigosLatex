Durante os testes que foram realizados sobre a ferramenta, notamos que o kura faz uso de uma documentação incompleta e inconsistente com a realidade do desenvolvimento de plug-ins, tornando o desenvolvimento e utilização da tecnologia complexa e desgastante para quem não tem experiência com OSGI.

Isto ocorre porque há pontos faltando nos tutoriais, como por exemplo no tutorial de instalação e configuração do kura no computador, no passo 4, o tutorial oficial afirmava que os erros no projeto desapareceriam após definir a plataforma alvo de execução. Porém isso poderia não acontecer caso o eclipse estivesse configurado para dar suporte no mínimo a versão 7 do java. Para que os erros sumam, a versão mínima do java configurada no eclipse, deve ser a versão 5.
Além disso, alguns tutoriais assumiram que o programador já possuía conhecimento prévio sobre alguns assuntos, como por exemplo no tutorial oficial “Usando o emulador: com o Iagent”, para executá-lo com sucesso, o usuário tinha que saber como funcionava tal processo no OSGI, porém o tutorial não indicava que esse conhecimento era necessário.

Depois deste e de outros testes realizados com o Kura, vimos que ele não é simples de ser usado, principalmente por quem nunca usou um servidor OSGi antes. Além de se tratar de uma nova tecnologia, o que diminuía a quantidade de material disponível na internet, para pesquisa. As informações que encontramos, estão desatualizadas, e não possuem foco, em um framework que torne mais fácil, o uso do padrão OSGI, como o kura. Porém é uma ferramenta interessante de se trabalhar, pois dá suporte ao desenvolvimento de plug-ins usando a linguagem de programação Java e possibilitando a criação de aplicações que deem suporte à IoT.

Após analisarmos a documentação oficial, resolvemos criar novos tutoriais baseados nos tutoriais oficiais. Nosso material foi testado e avaliado de acordo com os critérios que foram pré-definidos, ou adicionados durante o processo de análise. Após a realização dos testes foi provado que mesmo que o leitor não tenha um conhecimento profundo sobre OSGI e Internet das coisas, consegue executar com sucesso nossos tutoriais, tornando o Kura uma ferramenta útil para o desenvolvimento de plug-ins voltados para a internet das coisas. Além disso concluímos que:

\begin{enumerate}
  \item A raspberry pi é uma ótima plataforma para utilização do Kura.
  \item O kura é uma ferramenta com bom potencial para IoT, mesmo ainda em desenvolvimento. Apresenta alguns problemas.
  \item Para utilização do Kura, é necessário uma vasto conhecimento prévio sobre o padrão OSGI.
  \item A documentação assume que o usuário é especialista, pois contém pontos faltando, incorretos ou incompletos.
  \item O kura ainda precisa resolver alguns problemas para melhorar a experiência do usuário.
  \item Documentação precisa melhorar.
\end{enumerate}
