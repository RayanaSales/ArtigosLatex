Internet of Things (IoT) is the way things are connected and communicate with each other and with the user, through intelligent sensors and software that transmit data to a network. It is interesting to associate IoT with OSGI (Open Services Gateway Initiative) technology because it defines an architecture for the development and deployment of modular applications and libraries. With OSGI it is possible to build simple applications using implementations such as Kura.

\setlength{\parindent}{5ex} Since Kura is still in development, some difficulties were encountered during the tests that were performed with the tool, such as: incomplete, ambiguous and unintuitive documentation for the user. In the face of such problems, we then decide to test it and evaluate a quality of support for Kura gives the development of plug-ins focused on IoT. Our evaluation encompasses the installation process, the development environment, the execution environments, a documentation, and the tutorials. \par

For the development of this work, we have chosen some of the Kura tutorials, which we consider indispensable for the understanding and learning of technology, and we create new tutorials that are complementary and describe solutions to the most common problems. Such tutorials were tested by an external evaluator, who has knowledge about OSGI and development of plug-ins, was able to successfully execute our services. 