\documentclass[12pt]{article}

\usepackage{sbc-template}

\usepackage{graphicx,url}

\usepackage[brazil]{babel}
%\usepackage[latin1]{inputenc}
\usepackage[utf8]{inputenc}
% UTF-8 encoding is recommended by ShareLaTex


\sloppy

\title{Análise da ferramenta Kura com foco em internet das coisas}
\author{Rayana A. Sales\inst{1}, Paulo A. Guedes\inst{2} }
\address{Instituto Federal de Pernambuco de Recife
  (IFPE)\\
  Caixa Postal 15.064 -- 91.501-970 -- Recife -- PE -- Brasil
\nextinstitute
  Departamento de Tecnologia em Análise e Desenvolvimento de Sistemas -- IFPE\\
  Recife, PE.
\nextinstitute
  Departamento de Tecnologia em Análise e Desenvolvimento de Sistemas\\
  Instituto Federal de Pernambuco (IFPE) -- RECIPE, PE -- Brasil
  \email{ras@a.recife.ifpe.edu.br, paulo.guedes@recife.ifpe.edu.br}
}

\begin{document}

\maketitle

\begin{abstract}
\setlength{\parindent}{5ex}Kura is one of the tools that based on the Open Services Gateway Initiative (OSGI) technology, supports the development of Internet-based applications of things (IoT). Because it is still in development, its documentation still presents some difficulties, such as ambiguity and missing steps. We then decide to evaluate the quality of the support that Kura gives to the development of plug-ins, and complement the official tutorials with solutions to the most common problems. Our documentation was tested by an external evaluator, who, having little knowledge of OSGI, was able to execute the tutorials successfully.\par
\end{abstract}

\begin{resumo} 
% Qual é o problema? Por que o problema eh interessante? O que sua solução faz? Principais conclusões
Internet das coisas (IoT) é o modo como as coisas estão conectadas e se comunicam entre si e com o usuário, através de sensores inteligentes e softwares que transmitem dados para uma rede. É interessante associar IoT à tecnologia OSGI, pois ela define uma arquitetura para o desenvolvimento e implantação de aplicativos e bibliotecas modulares. Com OSGI é possível construir aplicações simples usando implementações como o Kura.

\setlength{\parindent}{5ex} Estando o Kura ainda em desenvolvimento, algumas dificuldades foram encontradas durante os testes que foram realizados com a ferramenta, tais como: documentação incompleta, ambígua e pouco intuitiva para o usuário. Diante tais problemas, resolvemos então testá-lo e avaliar a qualidade do suporte que o Kura dá ao desenvolvimento de plug-ins com foco em IoT. A nossa avaliação engloba o processo de instalação, o ambiente de desenvolvimento, os ambientes de execução, a documentação disponível, e os tutoriais. \par

Para o desenvolvimento deste trabalho, escolhemos alguns dos tutoriais do Kura, que julgamos indispensáveis para o entendimento e aprendizagem da tecnologia e criamos novos tutoriais que os complementam e descrevem possíveis soluções para os problemas mais comuns. Tais tutoriais foram testados por um avaliador externo, que possuindo poucos conhecimentos sobre OSGI e desenvolvimento de plug-ins, conseguiu executar com sucesso os nossos tutoriais. 
\end{resumo}

\section{Introdução}
% versão expandida do resumo. breve descrição de sua ideia. Os resultados principais (sem suspense em artigo)
IoT é um conceito tecnológico em que todos os objetos da vida cotidiana estariam conectados à internet, agindo de modo inteligente e sensorial. Hoje, milhões de dispositivos já estão conectados à internet com sensores, desafiando o conceito de computador. Assim, softwares e sensores estão controlando cada vez mais o que era feito antes apenas por seres humanos, com mais eficácia, conveniência e a custos reduzidos.

Apesar de ainda ser uma área pouco explorada na computação, a IoT já está presente no nosso cotidiano, em forma de relógios e óculos inteligentes, não sendo mais apenas uma ideia de um futuro distante. Um exemplo de IoT com qual já estamos bem familiarizados é o GPS, que monitora e processa os dados do tráfego em tempo real ajudando a gerenciar infraestruturas de transporte, avaliar as condições da estrada e aliviar o congestionamento.

Para a criação de aplicações para IoT, foram desenvolvidas várias ferramentas, uma delas é o Kura, feita pelo time responsável pelo desenvolvimento da IDE Eclipse. O Kura se trata de um servidor de aplicações baseados na tecnologia OSGI, que permite a criação e gerenciamento do ciclo de vida de plug-ins. O kura oferece também um ambiente de testes integrado ao ambiente de desenvolvimento e um shell similar ao shell do linux, onde é possível enviar comandos aos plug-ins.

% TRABALHOS RELACIONADOS
Para a realização deste projeto, foi analisado o trabalho de Vinicius Bevilaqua, onde foi realizado uma análise através de testes sobre três frameworks com foco no desenvolvimento em IoT: Eclipe Kura, The Thing box e Webiopi, afim de identificar uma alternativa que facilite e agilize o desenvolvimento para dispositivos inteligentes. Porém ao fim do trabalho, Vinicius concluiu que o projeto The Thing Box demonstrou o melhor resultado entre os três frameworks, além de apresentar uma proposta de desenvolvimento mais amigável para novos desenvolvedores. Vinicius também concluiu em seu trabalho, que o Kura é complexo e que foi projetado para desenvolvedores Java experientes com técnicas e termos avançados. Devido ao alto nível de complexidade, o autor não se aprofundou o quanto gostaríamos na ferramenta, focando mais na The Thing box.

Além do trabalho de Vinicius Bevilaqua, não foram encontrados trabalhos relacionados ao Kura. Devido a falta de material disponível para pesquisa, e aos problemas encontrados durante aos testes realizados com o Kura, resolvemos criar e disponibilizar na Internet, novos tutoriais baseados nos tutoriais oficiais do Kura, onde descrevemos possíveis soluções para problemas que são comumente encontrados. Ao final de cada tutorial, realizamos também uma avaliação, onde descrevemos o que aprendemos e nossas impressões.
 
% resultados principais.

Após submeter os nossos tutoriais a testes, com um avaliador externo que possuia pouco conhecimento sobre IoT, e nenhum sobre o Kura, provamos que o nosso material fornece uma base bem estruturada de passos, que guiam o programador inexperiente a conclusão dos tutoriais com sucesso.


\section{Revisão da literatura}
% Tem que informar o que existe e quais são as limitaçõese as extensões necessárias
% Ajuda o leitor a se familiarizar com o assunto
No contexto de IoT, uma tecnologia relevante é o padrão OSGI. Este padrão define um mecanismo que oferece suporte a plug-ins em Java, controlando sua instalação e o seu ciclo de vida. Os plug-ins criados em um servidor OSGI podem oferecer e consumir serviços entre si, bem como se comunicar \cite{IntroducaoOsgi}. Tais plug-ins podem construir aplicações que podem ser quebradas em vários módulos e que permita facilmente, gerenciar os cruzamentos entre eles. Gerenciar os módulos de forma independente, ou gerenciar um grupo de módulos que trabalham juntos \cite{OsgiVantagens}.

O padrão OSGI pode ser utilizado para o desenvolvimento de aplicações baseados em Internet das coisas e o Kura é um exemplo de ferramenta que dá suporte a tal desenvolvimento. Sendo um servidor de aplicações OSGI, o kura permite a criação de plug-ins usando a linguagem de programação java. Também permite o gerenciamento do ciclo de vida de plug-ins, e teste das soluções desenvolvidas usando um emulador que permite a execução em um ambiente de desenvolvimento. O kura também oferece serviços para conexão com portas seriais, GPS, bluetooth e nuvem.

O Kura é uma das ferramentas, baseada na tecnologia OSGI, que dá suporte ao desenvolvimento de soluções para a internet das coisas. O Kura se trata de um servidor OSGI com foco em IoT, que fornece uma plataforma de suporte a plug-ins já com vários plugins pré-definidos que são úteis na criação de sistemas para IoT.

O kura se encontra disponível para os sistemas operacionais Mac e Linux. O Linux é ao mesmo tempo um kernel (ou núcleo) e um sistema operacional que roda sobre ele. \cite{campos2006linux}. O Linux possui alguns programas interessantes tais como fswebcam e espeak. O espeak se trata de um sintetizador de voz para o Ubuntu, onde o usuário pode inserir entradas de textos e estas serão sintetizadas e reproduzidas pelo espeak. O espeak dá suporte a diversas línguas, inclusive o português. Outro programa disponível para o linux é o fswebcam, que se trata de um programa para tirar fotos. O programa recebe diversos parâmetros antes de tirar a foto, tais como informações sobre o tamanho do arquivo, formato, e tempo que o programa espera antes de tirar a foto. O local onde o arquivo será salvo, também pode ser passado como parâmetro.




\section{Metodologia}
% descreva: Solução proposta. Metodologia adotada para avaliar a solução. Modelos utilizados. Planejamento de experimentos (se houver). Não incluir resultados aqui.
%breve descrição do problema

Em busca de uma ferramenta que nos permitisse trabalhar com IoT de uma forma que fosse fácil implementar e reusar serviços, encontramos o Kura. Estudamos sobre ele, com foco nas partes: instalação, documentação, ambiente de desenvolvimento e ambiente de execução. Testamos as partes descritas e sentimos dificuldade, principalmente em nossos estudos realizados sob a documentação.

\subsection{Metodologia de análise e avaliação}

% como a analise foi realizada? como geramos a documentação?

Durante os nossos testes preliminares e a execução dos tutoriais disponíveis na api do Kura, adquirimos conhecimentos e então realizamos um estudo de casos, sobre a ferramenta. Identificamos tópicos importantes para a realização com sucesso dos tutoriais, que foram omitidos da documentação original e com isto, criamos novos tutoriais. Nosso material engloba pontos que são importantes para o entendimento do leitor, e que o ajudam a trabalhar de uma forma intuitiva, com plug-ins osgi.

\subsection{Experimento}

% como tudo isso foi testado?

Este projeto gerou uma coleção de documentos que precisaram ser testados, por um avaliador externo. O avaliador escolhido para os testes não tinha conhecimento profundo sobre IoT, nem sobre o Kura, pois era necessário provar que o nosso material é suficiente para o avaliador entender como o Kura funciona e trabalhar com ele.

Os testes que o avaliador deveria realizar foram divididos em três fases: Fase 1 - Testes da documentação, onde o avaliador seguiu os passos descritos em nossos tutoriais e registrou os resultados. Fase 2 - Testes dos artefatos gerados, onde o avaliador executou os códigos gerados ao decorrer deste projeto, e registrou se conseguiu instalar e executar com sucesso as aplicações. Fase 3 - Testes dos tutoriais não cobertos por este trabalho, onde o avaliador escolheu alguns tutoriais no site oficial do Kura, e os executou, a fim de verificar se após o processo, possuia conhecimento suficiente para trabalhar com o Kura de acordo com suas próprias necessidades.

\subsection{Definição e formatação dos testes}

Durante os testes referentes a Fase 1, onde foi analisada a documentação, o avaliador registrou:

\begin{enumerate}
  \item O tempo gasto em cada tutorial. Registrando hora de início e fim.
  \item Se concluiu ou não com sucesso o tutorial. Em caso negativo, registrou até onde conseguiu.
  \item Se o resultado observado foi o mesmo que o tutorial descrevia ou se houve alguma mudança.
  \item Se há alguma coisa faltando nos tutoriais.
  \item Suas dificuldades e impressões sobre os tutoriais.
\end{enumerate}

Durante os testes referentes a Fase 2, onde foram analisados aos artefatos gerados, o avaliador registrou, se conseguiu:

\begin{enumerate}
  \item Instalar as aplicações.
  \item Testar manualmente as aplicações, a fim de verificar se estão funcionando.
  \item Gerar os pacotes OSGI.
  \item Instalar na raspberry os pacotes gerados.
  \item Executar plugins que consomem serviços de tais aplicações previamente criadas.
\end{enumerate}

Durante os testes referentes a Fase 3, onde foi analisado o desempenho do avaliador ao executar os tutoriais não cobertos por este trabalho, o avaliador registrou quais tutoriais disponíveis na documentação oficial do Kura, escolheu executar e se conseguiu concluir com sucesso tais tutoriais, usando o conhecimento adquirido com os tutoriais anteriores.







\section{Resultados}
\subsection{E-speak e Fswebcam}
Como parte da nossa avaliação, foram escritos dois plug-ins para testar o nosso entendimento sobre a criação de plug-ins usando o kura, e se a ferramenta se comportava como esperado desenvolvendo outras aplicações, que não foram previstas nos tutoriais oficiais. Tais plug-ins desenvolvidos encapsulam as principais funcionalidades dos programas espeak e fswebcam, tais como sintetizar voz e tirar fotos, em um plug-in do kura, que pode ser reusado por outras aplicações.

Para realizar a síntese de voz, o plug-in recebe como parâmetro, o texto e o idioma no qual o texto deve ser reproduzido.
Para tirar foto, o plug-in recebe como parâmetro o nome do arquivo, e a quantidade de quadros que o programa deve esperar, antes de salvar a foto. Desta forma as funcionalidades dos programas são disponibilizadas como serviços do kura, e os serviços podem ser consumidos por outros plug ins.

\subsection{Avaliação dos artefatos gerados}

Para a realização dos testes, foi escolhido um menstrando em ciência da computação, do centro de informatica de pernambuco, que não possuia experiência com o desenvolvimento OSGI, e IoT. O aluno executou as três fases dos testes, descritas seção ~\ref{sec:metodologia} deste artigo.

Ao realizar os testes da Fase 1, onde foram analisadas a documentação gerada neste projeto, foi constatado que o aluno levou aproximadamente 40 minutos ao executar cada tutorial solicitado. O aluno registrou que conseguiu executar os tutoriais com sucesso e que os resultado observados após a execução dos tutoriais, foram os mesmos previstos. Porém o aluno documentou que nem todos os fluxos alternativos foram cobertos pelos tutoriais. Ao criar o primeiro plug-in o aluno constatou que ao importar os projetos do Kura em seu workspace pessoal, os erros não sumiram após configurar a versão do java para 1.5, como previsto no fluxo alternativo do tutorial. Foi preciso limpar e construir (clean and rebuild) o projeto.

Ao realizar os testes da Fase 2, onde foram analisadas os artefatos gerados neste projeto, foi constatado que o aluno conseguiu instalar, executar e gerar pacotes OSGI dos plug-ins desenvolvidos, para exportação. O aluno instalou os pacotes exportados em uma máquina virtual com o Kura, e conseguiu consumir com sucesso os serviços disponibilizados pelos plug-ins. Porém os testes não foram realizados também em uma raspberry, como esperado, devido a indisponibilidade do mesmo.

Ao realizar os testes da Fase 3, onde foi analisado o desempenho do aluno ao executar tutoriais não abordados neste projeto, foi constatado que o aluno conseguiu executar com sucesso o tutorial escolhido por ele: "Configuração da nuvem" \cite{CloudConfiguration}. Porém o aluno registrou que não há a certeza de que a configuração foi executada corretamente, pois o tutorial não aborda como testar o uso da nuvem, usando o Kura.

\subsection{Outros resultados}

Além dos resultados previamente descritos, outros resultados foram alcançados no decorrer deste projeto, são eles:

\begin{enumerate}
  \item Aprendizagem da tecnologia OSGI com o kura.
  \item Desenvolvimento de plug-ins como prova de conceito.
  \subitem Plug-ins baseados nos tutoriais oficiais.
  \subitem Fswebcam, Espeak.
  \item Revisão e criação de novos tutoriais comentados.
  \item Definição e formatação de testes para avaliação do ambiente.
\end{enumerate} 

\section{Conclusões}
Durante os testes que foram realizados sobre a ferramenta, notamos que o kura faz uso de uma documentação incompleta e inconsistente com a realidade do desenvolvimento de plug-ins, tornando o desenvolvimento e utilização da tecnologia complexa e desgastante para quem não tem experiência com OSGI.

Isto ocorre porque há pontos faltando nos tutoriais, como por exemplo no tutorial de instalação e configuração do kura no computador, no passo 4, o tutorial oficial afirmava que os erros no projeto desapareceriam após definir a plataforma alvo de execução. Porém isso poderia não acontecer caso o eclipse estivesse configurado para dar suporte no mínimo a versão 7 do java. Para que os erros sumam, a versão mínima do java configurada no eclipse, deve ser a versão 5.
Além disso, alguns tutoriais assumiram que o programador já possuía conhecimento prévio sobre alguns assuntos, como por exemplo no tutorial oficial “Usando o emulador: com o Iagent”, para executá-lo com sucesso, o usuário tinha que saber como funcionava tal processo no OSGI, porém o tutorial não indicava que esse conhecimento era necessário.

Depois deste e de outros testes realizados com o Kura, vimos que ele não é simples de ser usado, principalmente por quem nunca usou um servidor OSGi antes. Além de se tratar de uma nova tecnologia, o que diminuía a quantidade de material disponível na internet, para pesquisa. As informações que encontramos, estão desatualizadas, e não possuem foco, em um framework que torne mais fácil, o uso do padrão OSGI, como o kura. Porém é uma ferramenta interessante de se trabalhar, pois dá suporte ao desenvolvimento de plug-ins usando a linguagem de programação Java e possibilitando a criação de aplicações que deem suporte à IoT.

Após analisarmos a documentação oficial, resolvemos criar novos tutoriais baseados nos tutoriais oficiais. Nosso material foi testado e avaliado de acordo com os critérios que foram pré-definidos, ou adicionados durante o processo de análise. Após a realização dos testes foi provado que mesmo que o leitor não tenha um conhecimento profundo sobre OSGI e Internet das coisas, consegue executar com sucesso nossos tutoriais, tornando o Kura uma ferramenta útil para o desenvolvimento de plug-ins voltados para a internet das coisas. Além disso concluímos que:

\begin{enumerate}
  \item A raspberry pi é uma ótima plataforma para utilização do Kura.
  \item O kura é uma ferramenta com bom potencial para IoT, mesmo ainda em desenvolvimento. Apresenta alguns problemas.
  \item Para utilização do Kura, é necessário uma vasto conhecimento prévio sobre o padrão OSGI.
  \item A documentação assume que o usuário é especialista, pois contém pontos faltando, incorretos ou incompletos.
  \item O kura ainda precisa resolver alguns problemas para melhorar a experiência do usuário.
  \item Documentação precisa melhorar.
\end{enumerate}


\section{Agradecimentos}
Agradeço ao DGTI-IFPE pelo apoio técnico durante a execução do projeto. Agradeço também a Lizianne Priscila Marques Souto, pelos conhecimentos transmitidos de suma importância que tornaram possível a escrita deste artigo. E finalmente agradeço ao meu orientador Paulo Abadie Guedes, pela paciência ao guiar o projeto, e pelo profissionalismo que para mim, se tornou um exemplo a ser seguido.

\bibliographystyle{sbc}
\bibliography{sbc-template}

\end{document}
