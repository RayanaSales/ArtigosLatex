\documentclass[12pt]{article}

\usepackage{sbc-template}

\usepackage{graphicx,url}

\usepackage[brazil]{babel}
%\usepackage[latin1]{inputenc}
\usepackage[utf8]{inputenc}
% UTF-8 encoding is recommended by ShareLaTex


\sloppy

\title{Análise da ferramenta Kura com foco em internet das coisas}
\author{Rayana A. Sales\inst{1}, Paulo A. Guedes\inst{2} }
\address{Instituto Federal de Pernambuco de Recife
  (IFPE)\\
  Caixa Postal 15.064 -- 91.501-970 -- Recife -- PE -- Brasil
\nextinstitute
  Departamento de Tecnologia em Análise e Desenvolvimento de Sistemas -- IFPE\\
  Recife, PE.
\nextinstitute
  Departamento de Tecnologia em Análise e Desenvolvimento de Sistemas\\
  Instituto Federal de Pernambuco (IFPE) -- RECIPE, PE -- Brasil
  \email{ras@a.recife.ifpe.edu.br, paulo.guedes@recife.ifpe.edu.br}
}

\begin{document}

\maketitle

\begin{abstract}
Internet of Things (IoT) is the way things are connected and communicate with each other and with the user, through intelligent sensors and software that transmit data to a network. It is interesting to associate IoT with OSGI technology because it defines an architecture for the development and deployment of modular applications and libraries. With OSGI it is possible to build simple applications using implementations such as Kura.

\setlength{\parindent}{5ex} Since Kura is still in development, some difficulties were encountered during the tests that were performed with the tool, such as: incomplete, ambiguous and unintuitive documentation for the user. In the face of such problems, we then decide to test it and evaluate a quality of support for Kura gives the development of plug-ins focused on IoT. Our evaluation encompasses the installation process, the development environment, the execution environments, a documentation, and the tutorials. \par

For the development of this work, we have chosen some of the Kura tutorials, which we consider indispensable for the understanding and learning of technology, and we create new tutorials that are complementary and describe solutions to the most common problems. Such tutorials were tested by an external evaluator, who has knowledge about OSGI and development of plug-ins, was able to successfully execute our services. 
\end{abstract}

\begin{resumo}
% Qual é o problema? Por que o problema eh interessante? O que sua solução faz? Principais conclusões
Internet das coisas (IoT) é o modo como as coisas estão conectadas e se comunicam entre si e com o usuário, através de sensores inteligentes e softwares que transmitem dados para uma rede. É interessante associar IoT à tecnologia OSGI, pois ela define uma arquitetura para o desenvolvimento e implantação de aplicativos e bibliotecas modulares. Com OSGI é possível construir aplicações simples usando implementações como o Kura.

Estando o Kura ainda em desenvolvimento, algumas dificuldades foram encontradas durante os testes que foram realizados com a ferramenta, tais como: documentação incompleta, ambígua e pouco intuitiva para o usuário. Diante tais problemas, resolvemos então testá-lo e avaliar a qualidade do suporte que o Kura dá ao desenvolvimento de plug-ins com foco em IoT. A nossa avaliação engloba o processo de instalação, o ambiente de desenvolvimento, os ambientes de execução, a documentação disponível, e os tutoriais.

Para o desenvolvimento deste trabalho, escolhemos alguns dos tutoriais do Kura, que julgamos indispensáveis para o entendimento e aprendizagem da tecnologia e criamos novos tutoriais que os complementam e descrevem possíveis soluções para os problemas mais comuns. Tais tutoriais foram testados por um avaliador externo, que possuindo poucos conhecimentos sobre OSGI e desenvolvimento de plug-ins, conseguiu executar com sucesso os nossos tutoriais.
\end{resumo}

\section{Introdução}
% versão expandida do resumo. breve descrição de sua ideia. Os resultados principais (sem suspense em artigo)
IoT é um conceito tecnológico em que todos os objetos da vida cotidiana estariam conectados à internet, agindo de modo inteligente e sensorial \cite{IoTSignificado}. Hoje, milhões de dispositivos já estão conectados à internet com sensores, desafiando o conceito de computador. Assim, softwares e sensores estão controlando cada vez mais o que era feito antes apenas por seres humanos, com mais eficácia, conveniência e a custos reduzidos.

Apesar de ainda ser uma área pouco explorada na computação, a IoT já está presente no nosso cotidiano, em forma de relógios e óculos inteligentes, não sendo mais apenas uma ideia de um futuro distante. Um exemplo de IoT com qual já estamos bem familiarizados é o GPS, que monitora e processa os dados do tráfego em tempo real ajudando a gerenciar infraestruturas de transporte, avaliar as condições da estrada e aliviar o congestionamento.

Para a criação de aplicações para IoT, foram desenvolvidas várias ferramentas, uma delas é o Kura, feita pelo time responsável pelo desenvolvimento da IDE Eclipse. O Kura se trata de um servidor de aplicações baseados na tecnologia OSGI, que permite a criação e gerenciamento do ciclo de vida de plug-ins. O kura oferece também um ambiente de testes integrado ao ambiente de desenvolvimento e um shell similar ao shell do linux, onde é possível enviar comandos aos plug-ins.

% problema e objtivo geral

Apesar de se tratar de uma ferramenta bastante poderosa, o Kura ainda se encontra em desenvolvimento. Sua documentação assume que o leitor possui um vasto conhecimento sobre desenvolvimento de plug-ins, e omite passos dos tutoriais tornando o aprendizado complexo e exaustivo. Baseando-se em tal empecilho, o nosso objetivo geral é tornar o uso da ferramenta mais fácil, disponibilizando novos tutoriais, provas de conceitos e avaliações, que visam exemplificar o uso do Kura, e compartilhar um pouco da nossa experiência e do que aprendemos.

% objetivos gerais e especificos

Além do objetivo descrito acima, também temos como objetivo:

\begin{enumerate}
\item Definir uma plataforma de execução para os testes e para avaliação do sistema, visto que uma parte dos testes serão efetuados no emulador do Kura e a outra na raspberry pi.
\item Ler a documentação do kura, instalar o ambiente de desenvolvimento e executar uma prova de conceito para provar que o ambiente de desenvolvimento e de execução funcionam bem.
\item Definir uma estratégia de avaliação para a plataforma Kura.
\item Implementar uma coleção de plugins que exercitem os serviços do ambiente Kura, para avaliar sua adequação no suporte a aplicações de IoT.
\item Avaliar a plataforma de acordo com os critérios pré-definidos, ou adicionados durante o processo.
\end{enumerate}



\section{Revisão da literatura}
% Tem que informar o que existe e quais são as limitaçõese as extensões necessárias
% Ajuda o leitor a se familiarizar com o assunto
No contexto de IoT, uma tecnologia relevante é o padrão OSGI. Este padrão define um mecanismo que oferece suporte a plug-ins em Java, controlando sua instalação e o seu ciclo de vida. Os plug-ins criados em um servidor OSGI podem oferecer e consumir serviços entre si, bem como se comunicar \cite{IntroducaoOsgi}. Tais plug-ins podem construir aplicações que podem ser quebradas em vários módulos e que permita facilmente, gerenciar os cruzamentos entre eles. Gerenciar os módulos de forma independente, ou gerenciar um grupo de módulos que trabalham juntos \cite{OsgiVantagens}.

O padrão OSGI pode ser utilizado para o desenvolvimento de aplicações baseados em Internet das coisas e o Kura é um exemplo de ferramenta que dá suporte a tal desenvolvimento. Sendo um servidor de aplicações OSGI, o kura permite a criação de plug-ins usando a linguagem de programação java. Também permite o gerenciamento do ciclo de vida de plug-ins, e teste das soluções desenvolvidas usando um emulador que permite a execução em um ambiente de desenvolvimento. O kura também oferece serviços para conexão com portas seriais, GPS, bluetooth e nuvem.

O Kura uma das ferramentas, baseada na tecnologia OSGI, que dá suporte ao desenvolvimento de soluções para a internet das coisas. O Kura se trata de um servidor OSGI com foco em IoT, que fornece uma plataforma de suporte a plug-ins já com vários plugins pré-definidos que são úteis na criação de sistemas para IoT.

O kura se encontra disponível para os sistemas operacionais Mac e Linux. O Linux é ao mesmo tempo um kernel (ou núcleo) e um sistema operacional que roda sobre ele. \cite{campos2006linux}. O Linux possui alguns programas interessantes tais como fswebcam e espeak. O espeak se trata de um sintetizador de voz para o Ubuntu, onde o usuário pode inserir entradas de textos e estas serão sintetizadas e reproduzidas pelo espeak. O espeak dá suporte a diversas línguas, inclusive o português. Outro programa disponível para o linux é o fswebcam, que se trata de um programa para tirar fotos. O programa recebe diversos parâmetros antes de tirar a foto, tais como informações sobre o tamanho do arquivo, formato, e tempo que o programa espera antes de tirar a foto. O local onde o arquivo será salvo, também pode ser passado como parâmetro.




\section{Metodologia}
% descreva: Solução proposta. Metodologia adotada para avaliar a solução. Modelos utilizados. Planejamento de experimentos (se houver). Não incluir resultados aqui.
%breve descrição do problema

Em busca de uma ferramenta que nos permitisse trabalhar com IoT de uma forma que fosse fácil implementar e reusar serviços, encontramos o Kura. Estudamos sobre ele, com foco nas partes: instalação, documentação, ambiente de desenvolvimento e ambiente de execução. Testamos as partes descritas e sentimos dificuldade, principalmente em nossos estudos realizados sob a documentação.

\subsection{Metodologia de análise e avaliação}\label{sec:metodologia}

% como a analise foi realizada? como geramos a documentação?

Durante os nossos testes preliminares e a execução dos tutoriais disponíveis na api do Kura, adquirimos conhecimentos e então realizamos um estudo de casos, sobre a ferramenta. Identificamos tópicos importantes para a realização com sucesso dos tutoriais, que foram omitidos da documentação original e com isto, criamos novos tutoriais. Nosso material engloba pontos que são importantes para o entendimento do leitor, e que o ajudam a trabalhar de uma forma intuitiva, com plug-ins osgi. A seguir, são descritos os tutoriais oficiais \cite{KuraDocumentation}, que foram abordados neste projeto:

\begin{enumerate}
  \item Instalação e configuração do kura no computador.
  \item Criando o primeiro plug-in.
  \item Usando o emulador: com o Iagent.
  \item Parando plug-in permanentemente.
  \item Exportando e importando serviço.
  \item Exportando e importando serviço de forma otimizada.
  \item Monitoração de plug-in.
  \item Watchdog
\end{enumerate}

\subsection{Experimento}

% como tudo isso foi testado?

Este projeto gerou uma coleção de documentos que precisaram ser testados, por um avaliador externo. O avaliador escolhido para os testes não tinha conhecimento profundo sobre IoT, nem sobre o Kura, pois era necessário provar que o nosso material é suficiente para o avaliador entender como o Kura funciona e trabalhar com ele.

Os testes que o avaliador deveria realizar foram divididos em três fases: Fase 1 - Testes da documentação, onde o avaliador seguiu os passos descritos em nossos tutoriais e registrou os resultados. Fase 2 - Testes dos artefatos gerados, onde o avaliador executou os códigos gerados ao decorrer deste projeto, e registrou se conseguiu instalar e executar com sucesso as aplicações. Fase 3 - Testes dos tutoriais não cobertos por este trabalho, onde o avaliador escolheu alguns tutoriais no site oficial do Kura, e os executou, a fim de verificar se após o processo, possuia conhecimento suficiente para trabalhar com o Kura de acordo com suas próprias necessidades.

\subsection{Definição e formatação dos testes}

Durante os testes referentes a Fase 1, onde foi analisada a documentação, o avaliador registrou:

\begin{enumerate}
  \item O tempo gasto em cada tutorial. Registrando hora de início e fim.
  \item Se concluiu ou não com sucesso o tutorial. Em caso negativo, registrou até onde conseguiu.
  \item Se o resultado observado foi o mesmo que o tutorial descrevia ou se houve alguma mudança.
  \item Se há alguma coisa faltando nos tutoriais.
  \item Suas dificuldades e impressões sobre os tutoriais.
\end{enumerate}
Durante os testes referentes a Fase 2, onde foram analisados aos artefatos gerados, o avaliador registrou, se conseguiu:

\begin{enumerate}
  \item Instalar as aplicações.
  \item Testar manualmente as aplicações, a fim de verificar se estão funcionando.
  \item Gerar os pacotes OSGI.
  \item Instalar na raspberry os pacotes gerados.
  \item Executar plugins que consomem serviços de tais aplicações previamente criadas.
\end{enumerate}
Durante os testes referentes a Fase 3, onde foi analisado o desempenho do avaliador ao executar os tutoriais não cobertos por este trabalho, o avaliador registrou quais tutoriais disponíveis na documentação oficial do Kura, escolheu executar e se conseguiu concluir com sucesso tais tutoriais, usando o conhecimento adquirido com os tutoriais anteriores.







\section{Resultados}
\subsection{Avaliação dos artefatos gerados}

\section{Conclusões}
Durante os testes que foram realizados sobre a ferramenta, notamos que o kura faz uso de uma documentação incompleta e inconsistente com a realidade do desenvolvimento de plug ins, tornando o desenvolvimento e utilização da tecnologia complexa e desgastante para quem não tem experiência com OSGI.

Isto ocorre porque há pontos faltando nos tutoriais como por exemplo, no tutorial de instalação e configuração do kura no computador, no passo 4, o tutorial oficial afirmava que os erros no projeto desapareceriam após definir a plataforma alvo de execução. Porém isso poderia não acontecer caso o eclipse estivesse configurado para dar suporte no mínimo a versão 7 do java. Para que os erros sumam, a versão mínima do java configurada no eclipse, deve ser a versão 5.
Além disso, alguns tutoriais assumiram que o programador já possuía conhecimento prévio sobre alguns assuntos, como por exemplo no tutorial oficial “Usando o emulador: com o Iagent”, para executá-lo com sucesso, o usuário tinha que saber como funcionava tal processo no OSGI, porém o tutorial não indicava que esse conhecimento era necessário.

Depois deste e de outros testes realizados com o Kura, vimos que ele não é simples de ser usado, principalmente por quem nunca usou um servidor OSGi antes. Além de se tratar de uma nova tecnologia, o que diminuía a quantidade de material disponível na internet, para pesquisa. As informações que encontramos, estão desatualizadas, e não possuem foco, em um framework que torne mais fácil, o uso do padrão OSGI, como o kura. Porém é uma ferramenta interessante de se trabalhar, pois dá suporte ao desenvolvimento de plug-ins usando a linguagem de programação Java e possibilitando a criação de aplicações que deem suporte à IoT.

Após analisarmos a documentação oficial, resolvemos criar novos tutoriais baseados nos tutoriais oficiais. Nosso material foi testado e avaliado de acordo com os critérios que foram pré-definidos, ou adicionados durante o processo de análise. Após a realização dos testes foi provado que mesmo que o leitor não tenha um conhecimento profundo sobre OSGI e Internet das coisas, consegue executar com sucesso nossos tutoriais, tornando o Kura uma ferramenta útil para o desenvolvimento de plug-ins voltados para a internet das coisas. Além disso concluímos que:

\begin{enumerate}
  \item A raspberry pi é uma ótima plataforma para utilização do Kura.
  \item O kura é uma ferramenta com bom potencial para IoT, mesmo ainda em desenvolvimento. Apresenta alguns problemas.
  \item Para utilização do Kura, é necessário uma vasto conhecimento prévio sobre o padrão OSGI.
  \item A documentação assume que o usuário é especialista, pois contém pontos faltando, incorretos ou incompletos.
  \item O kura ainda precisa resolver alguns problemas para melhorar a experiência do usuário.
  \item Documentação precisa melhorar.
\end{enumerate}


\section{Agradecimentos}
Agradeço ao DGTI-IFPE pelo apoio técnico durante a execução do projeto. Agradeço também a Lizianne Priscila Marques Souto, pelos conhecimentos transmitidos de suma importância que tornaram possível a escrita deste artigo. E finalmente agradeço ao meu orientador Paulo Abadie Guedes, pela paciência ao guiar o projeto, e pelo profissionalismo que para mim, se tornou um exemplo a ser seguido.

\bibliographystyle{sbc}
\bibliography{sbc-template}

\end{document}
