IoT é um conceito tecnológico em que todos os objetos da vida cotidiana estariam conectados à internet, agindo de modo inteligente e sensorial \cite{IoTSignificado}. Hoje, milhões de dispositivos já estão conectados à internet com sensores, desafiando o conceito de computador. Assim, softwares e sensores estão controlando cada vez mais o que era feito antes apenas por seres humanos, com mais eficácia, conveniência e a custos reduzidos.

Apesar de ainda ser uma área pouco explorada na computação, a IoT já está presente no nosso cotidiano, em forma de relógios e óculos inteligentes, não sendo mais apenas uma ideia de um futuro distante. Um exemplo de IoT com qual já estamos bem familiarizados é o GPS, que monitora e processa os dados do tráfego em tempo real ajudando a gerenciar infraestruturas de transporte, avaliar as condições da estrada e aliviar o congestionamento.

Para a criação de aplicações para IoT, foram desenvolvidas várias ferramentas, uma delas é o Kura, feita pelo time responsável pelo desenvolvimento da IDE Eclipse. O Kura se trata de um servidor de aplicações baseados na tecnologia OSGI, que permite a criação e gerenciamento do ciclo de vida de plug-ins. O kura oferece também um ambiente de testes integrado ao ambiente de desenvolvimento e um shell similar ao shell do linux, onde é possível enviar comandos aos plug-ins.

% problema e objtivo geral

Apesar de se tratar de uma ferramenta bastante poderosa, o Kura ainda se encontra em desenvolvimento. Sua documentação assume que o leitor possui um vasto conhecimento sobre desenvolvimento de plug-ins, e omite passos dos tutoriais tornando o aprendizado complexo e exaustivo. Baseando-se em tal empecilho, o nosso objetivo geral é tornar o uso da ferramenta mais fácil, disponibilizando novos tutoriais, provas de conceitos e avaliações, que visam exemplificar o uso do Kura, e compartilhar um pouco da nossa experiência e do que aprendemos.

% objetivos gerais e especificos

Além do objetivo descrito acima, também temos como objetivo:

\begin{enumerate}
\item Definir uma plataforma de execução para os testes e para avaliação do sistema, visto que uma parte dos testes serão efetuados no emulador do Kura e a outra na raspberry pi.
\item Ler a documentação do kura, instalar o ambiente de desenvolvimento e executar uma prova de conceito para provar que o ambiente de desenvolvimento e de execução funcionam bem.
\item Definir uma estratégia de avaliação para a plataforma Kura.
\item Implementar uma coleção de plugins que exercitem os serviços do ambiente Kura, para avaliar sua adequação no suporte a aplicações de IoT.
\item Avaliar a plataforma de acordo com os critérios pré-definidos, ou adicionados durante o processo.
\end{enumerate}

