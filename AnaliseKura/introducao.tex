IoT é um conceito tecnológico em que todos os objetos da vida cotidiana estariam conectados à internet, agindo de modo inteligente e sensorial \cite{IoTSignificado}. Hoje, milhões de dispositivos já estão conectados à internet com sensores, desafiando o conceito de computador. Assim, softwares e sensores estão controlando cada vez mais o que era feito antes apenas por seres humanos, com mais eficácia, conveniência e a custos reduzidos.

Apesar de ainda ser uma área pouco explorada na computação, a IoT já está presente no nosso cotidiano, em forma de relógios e óculos inteligentes, não sendo mais apenas uma ideia de um futuro distante. Um exemplo de IoT com qual já estamos bem familiarizados é o GPS, que monitora e processa os dados do tráfego em tempo real ajudando a gerenciar infraestruturas de transporte, avaliar as condições da estrada e aliviar o congestionamento.

Para a criação de aplicações para IoT, foram desenvolvidas várias ferramentas, uma delas é o Kura, feita pelo time responsável pelo desenvolvimento da IDE Eclipse. O Kura se trata de um servidor de aplicações baseados na tecnologia OSGI, que permite a criação e gerenciamento do ciclo de vida de plug-ins. O kura oferece também um ambiente de testes integrado ao ambiente de desenvolvimento e um shell similar ao shell do linux, onde é possível enviar comandos aos plug-ins.

% TRABALHOS RELACIONADOS
Para a realização deste projeto, foi analisado o trabalho de \cite{bevilaqua2016analise}, onde foi realizado uma análise através de testes sobre três frameworks com foco no desenvolvimento em IoT: Eclipe Kura, The Thing box e Webiopi, afim de identificar uma alternativa que facilite e agilize o desenvolvimento para dispositivos inteligentes. Porém ao fim do trabalho, Bevilaqua concluiu que o projeto The Thing Box demonstrou o melhor resultado entre os três frameworks, além de apresentar uma proposta de desenvolvimento mais amigável para novos desenvolvedores. Bevilaqua também concluiu em seu trabalho, que o Kura é complexo e que foi projetado para desenvolvedores Java experientes com técnicas e termos avançados. Devido ao alto nível de complexidade, o autor não se aprofundou o quanto gostaríamos na ferramenta, focando mais na The Thing box.

Além do trabalho de Bevilaqua, não foram encontrados trabalhos relacionados ao Kura. Devido a falta de material disponível para pesquisa, e aos problemas encontrados durante aos testes realizados com o Kura, resolvemos criar e disponibilizar na Internet, novos tutoriais baseados nos tutoriais oficiais do Kura, onde descrevemos possíveis soluções para problemas que são comumente encontrados. Ao final de cada tutorial, realizamos também uma avaliação, onde descrevemos o que aprendemos e nossas impressões.

% resultados principais.

Após submeter os nossos tutoriais a testes, com um avaliador externo que possuia pouco conhecimento sobre IoT, e nenhum sobre o Kura, provamos que o nosso material fornece uma base bem estruturada de passos, que guiam o programador inexperiente a conclusão dos tutoriais com sucesso.
