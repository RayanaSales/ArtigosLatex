\subsection{E-speak e Fswebcam}
Como parte da nossa avaliação, foram escritos dois plug-ins para testar o nosso entendimento sobre a criação de plug-ins usando o kura, e se a ferramenta se comportava como esperado desenvolvendo outras aplicações, que não foram previstas nos tutoriais oficiais. Tais plug-ins desenvolvidos encapsulam as principais funcionalidades dos programas espeak e fswebcam, tais como sintetizar voz e tirar fotos, em um plug-in do kura, que pode ser reusado por outras aplicações.

Para realizar a síntese de voz, o plug-in recebe como parâmetro, o texto e o idioma no qual o texto deve ser reproduzido.
Para tirar foto, o plug-in recebe como parâmetro o nome do arquivo, e a quantidade de quadros que o programa deve esperar, antes de salvar a foto. Desta forma as funcionalidades dos programas são disponibilizadas como serviços do kura, e os serviços podem ser consumidos por outros plug ins.

\subsection{Avaliação dos artefatos gerados}

Para a realização dos testes, foi escolhido um menstrando em ciência da computação, do centro de informatica de pernambuco, que não possuia experiência com o desenvolvimento OSGI, e IoT. O aluno executou as três fases dos testes, descritas seção ~\ref{sec:metodologia} deste artigo.

Ao realizar os testes da Fase 1, onde foram analisadas a documentação gerada neste projeto, foi constatado que o aluno levou aproximadamente 40 minutos ao executar cada tutorial solicitado. O aluno registrou que conseguiu executar os tutoriais com sucesso e que os resultado observados após a execução dos tutoriais, foram os mesmos previstos. Porém o aluno documentou que nem todos os fluxos alternativos foram cobertos pelos tutoriais. Ao criar o primeiro plug-in o aluno constatou que ao importar os projetos do Kura em seu workspace pessoal, os erros não sumiram após configurar a versão do java para 1.5, como previsto no fluxo alternativo do tutorial. Foi preciso limpar e construir (clean and rebuild) o projeto.

Ao realizar os testes da Fase 2, onde foram analisadas os artefatos gerados neste projeto, foi constatado que o aluno conseguiu instalar, executar e gerar pacotes OSGI dos plug-ins desenvolvidos, para exportação. O aluno instalou os pacotes exportados em uma máquina virtual com o Kura, e conseguiu consumir com sucesso os serviços disponibilizados pelos plug-ins. Porém os testes não foram realizados também em uma raspberry, como esperado, devido a indisponibilidade do mesmo.

Ao realizar os testes da Fase 3, onde foi analisado o desempenho do aluno ao executar tutoriais não abordados neste projeto, foi constatado que o aluno conseguiu executar com sucesso o tutorial escolhido por ele: "Configuração da nuvem" \cite{CloudConfiguration}. Porém o aluno registrou que não há a certeza de que a configuração foi executada corretamente, pois o tutorial não aborda como testar o uso da nuvem, usando o Kura.

\subsection{Discussão dos artefatos envolvendo trabalhos relacionados}

% TRABALHOS RELACIONADOS
Para a realização deste projeto, foi analisado o trabalho de \cite{bevilaqua2016analise}, onde foi realizado uma análise através de testes sobre três frameworks com foco no desenvolvimento em IoT: Eclipe Kura, The Thing box e Webiopi, afim de identificar uma alternativa que facilite e agilize o desenvolvimento para dispositivos inteligentes. Porém ao fim do trabalho, Bevilaqua concluiu que o projeto The Thing Box demonstrou o melhor resultado entre os três frameworks, além de apresentar uma proposta de desenvolvimento mais amigável para novos desenvolvedores. Bevilaqua também concluiu em seu trabalho, que o Kura é complexo e que foi projetado para desenvolvedores Java experientes com técnicas e termos avançados. Devido ao alto nível de complexidade, o autor não se aprofundou o quanto gostaríamos na ferramenta, focando mais na The Thing box.

Além do trabalho de Bevilaqua, não foram encontrados trabalhos relacionados ao Kura. Devido a falta de material disponível para pesquisa, e aos problemas encontrados durante aos testes realizados com o Kura, resolvemos criar e disponibilizar na Internet, novos tutoriais baseados nos tutoriais oficiais do Kura, onde descrevemos possíveis soluções para problemas que são comumente encontrados. Ao final de cada tutorial, realizamos também uma avaliação, onde descrevemos o que aprendemos e nossas impressões.

\subsection{Outros resultados}

% resultados principais.
Após submeter os nossos tutoriais a testes, com um avaliador externo que possuia pouco conhecimento sobre IoT, e nenhum sobre o Kura, provamos que o nosso material fornece uma base bem estruturada de passos, que guiam o programador inexperiente a conclusão dos tutoriais com sucesso. Além deste resultado, outros também foram alcançados no decorrer deste projeto, são eles:

\begin{enumerate}
  \item Aprendizagem da tecnologia OSGI com o kura.
  \item Desenvolvimento de plug-ins como prova de conceito.
  \subitem Plug-ins baseados nos tutoriais oficiais.
  \subitem Fswebcam, Espeak.
  \item Revisão e criação de novos tutoriais comentados.
  \item Definição e formatação de testes para avaliação do ambiente.
\end{enumerate} 