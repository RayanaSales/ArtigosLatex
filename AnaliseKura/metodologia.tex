%breve descrição do problema

Em busca de uma ferramenta que nos permitisse trabalhar com IoT de uma forma que fosse fácil implementar e reusar serviços, encontramos o Kura. Estudamos sobre ele, com foco nas partes: instalação, documentação, ambiente de desenvolvimento e ambiente de execução. Testamos as partes descritas e sentimos dificuldade, principalmente em nossos estudos realizados sob a documentação.

\subsection{Metodologia de análise e avaliação}\label{sec:metodologia}

% como a analise foi realizada? como geramos a documentação?

Durante os nossos testes preliminares e a execução dos tutoriais disponíveis na api do Kura, adquirimos conhecimentos e então realizamos um estudo de casos, sobre a ferramenta. Identificamos tópicos importantes para a realização com sucesso dos tutoriais, que foram omitidos da documentação original e com isto, criamos novos tutoriais. Nosso material engloba pontos que são importantes para o entendimento do leitor, e que o ajudam a trabalhar de uma forma intuitiva, com plug-ins OSGI. A seguir, são descritos os tutoriais oficiais \cite{KuraDocumentation}, que foram abordados neste projeto:

\begin{enumerate}
  \item Instalação e configuração do kura no computador.
  \item Criando o primeiro plug-in.
  \item Usando o emulador: com o Iagent.
  \item Parando plug-in permanentemente.
  \item Exportando e importando serviço.
  \item Exportando e importando serviço de forma otimizada.
  \item Monitoração de plug-in.
  \item Watchdog
\end{enumerate}

\subsection{Experimento}

% como tudo isso foi testado?

Este projeto gerou uma coleção de documentos que precisaram ser testados, por um avaliador externo. O avaliador escolhido para os testes não tinha conhecimento profundo sobre IoT, nem sobre o Kura, pois era necessário provar que o nosso material é suficiente para o avaliador entender como o Kura funciona e trabalhar com ele.

Os testes que o avaliador deveria realizar foram divididos em três fases: Fase 1 - Testes da documentação, onde o avaliador seguiu os passos descritos em nossos tutoriais e registrou os resultados. Fase 2 - Testes dos artefatos gerados, onde o avaliador executou os códigos gerados ao decorrer deste projeto, e registrou se conseguiu instalar e executar com sucesso as aplicações. Fase 3 - Testes dos tutoriais não cobertos por este trabalho, onde o avaliador escolheu alguns tutoriais no site oficial do Kura, e os executou, a fim de verificar se após o processo, possuia conhecimento suficiente para trabalhar com o Kura de acordo com suas próprias necessidades.

\subsection{Definição e formatação dos testes}

Durante os testes referentes a Fase 1, onde foi analisada a documentação, o avaliador registrou:

\begin{enumerate}
  \item O tempo gasto em cada tutorial. Registrando hora de início e fim.
  \item Se concluiu ou não com sucesso o tutorial. Em caso negativo, registrou até onde conseguiu.
  \item Se o resultado observado foi o mesmo que o tutorial descrevia ou se houve alguma mudança.
  \item Se há alguma coisa faltando nos tutoriais.
  \item Suas dificuldades e impressões sobre os tutoriais.
\end{enumerate}
Durante os testes referentes a Fase 2, onde foram analisados aos artefatos gerados, o avaliador registrou, se conseguiu:

\begin{enumerate}
  \item Instalar as aplicações.
  \item Testar manualmente as aplicações, a fim de verificar se estão funcionando.
  \item Gerar os pacotes OSGI.
  \item Instalar na raspberry os pacotes gerados.
  \item Executar plugins que consomem serviços de tais aplicações previamente criadas.
\end{enumerate}
Durante os testes referentes a Fase 3, onde foi analisado o desempenho do avaliador ao executar os tutoriais não cobertos por este trabalho, o avaliador registrou quais tutoriais disponíveis na documentação oficial do Kura, escolheu executar e se conseguiu concluir com sucesso tais tutoriais, usando o conhecimento adquirido com os tutoriais anteriores.





