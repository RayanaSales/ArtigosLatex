Internet das coisas (IoT) é o modo como as coisas estão conectadas e se comunicam entre si e com o usuário, através de sensores inteligentes e softwares que transmitem dados para uma rede. É interessante associar IoT à tecnologia OSGI (Open Services Gateway Initiative), pois ela define uma arquitetura para o desenvolvimento e implantação de aplicativos e bibliotecas modulares. Com OSGI é possível construir aplicações simples usando implementações como o Kura.

\setlength{\parindent}{5ex} Estando o Kura ainda em desenvolvimento, algumas dificuldades foram encontradas durante os testes que foram realizados com a ferramenta, tais como: documentação incompleta, ambígua e pouco intuitiva para o usuário. Diante tais problemas, resolvemos então testá-lo e avaliar a qualidade do suporte que o Kura dá ao desenvolvimento de plug-ins com foco em IoT. A nossa avaliação engloba o processo de instalação, o ambiente de desenvolvimento, os ambientes de execução, a documentação disponível, e os tutoriais. \par

Para o desenvolvimento deste trabalho, escolhemos alguns dos tutoriais do Kura, que julgamos indispensáveis para o entendimento e aprendizagem da tecnologia e criamos novos tutoriais que os complementam e descrevem possíveis soluções para os problemas mais comuns. Tais tutoriais foram testados por um avaliador externo, que possuindo poucos conhecimentos sobre OSGI e desenvolvimento de plug-ins, conseguiu executar com sucesso os nossos tutoriais. 